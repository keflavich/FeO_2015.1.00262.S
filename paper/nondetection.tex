\documentclass[twocolumn]{aastex61}
\usepackage{datetime} % TODO: remove this after submission
%\documentclass[defaultstyle,11pt]{thesis}
%\documentclass[]{report}
%\documentclass[]{article}
%\usepackage{aastex_hack}
%\usepackage{deluxetable}
%\documentclass[preprint]{aastex}
%\documentclass{aa}

\newcommand{\titlerunning}[1]{\shorttitle{#1}}
\newcommand{\authorrunning}[1]{\shortauthors{#1}}

\newcommand*\inst[1]{\unskip\hbox{\@textsuperscript{\normalfont$#1$}}}

%\newcount\aa@nbinstitutes
%
%\newcounter{aa@institutecnt}

\newcommand*\institute[1]{
  \begingroup
    \let\and\relax
    \renewcommand*\inst[1]{}%
    \renewcommand*\thanks[1]{}%
    \renewcommand*\email[1]{}%
    %\let\@@protect\protect
    %\let\protect\@unexpandable@protect
    %\global\aa@nbinstitutes \z@
    %\expandafter\aa@cntinstitutes\aa@institute\and\aa@nil\and
    %\restore@protect
  \endgroup
  \newcommand{\institutions}{#1}
}%

\let\oldarcsec\arcsec
\renewcommand\arcsec{\oldarcsec\xspace}%

%\renewcommand{\abstract}[1]{
%\begin{abstract}
%    #1
%\end{abstract}
%}

%\renewcommand\ion[2]{#1$\;${%
%\ifx\@currsize\normalsize\small \else
%\ifx\@currsize\small\footnotesize \else
%\ifx\@currsize\footnotesize\scriptsize \else
%\ifx\@currsize\scriptsize\tiny \else
%\ifx\@currsize\large\normalsize \else
%\ifx\@currsize\Large\large
%\fi\fi\fi\fi\fi\fi
%\rmfamily\@Roman{#2}}\relax}% 
%
%\renewcommand{ion}[2]{#1}{#2}

\renewcommand{\ion}[2]{\textup{#1\,\textsc{\lowercase{#2}}}}

%\newcommand{\uchii}{\ensuremath{\mathrm{\ion{UCH}{2}}}\xspace}
%\newcommand{\UCHII}{\ensuremath{\mathrm{\ion{UCH}{2}}}\xspace}
%\newcommand{\hchii}{\ensuremath{\mathrm{\ion{HCH}{2}}}\xspace}
%\newcommand{\HCHII}{\ensuremath{\mathrm{\ion{HCH}{2}}}\xspace}
%\newcommand{\hii}  {\ensuremath{\mathrm{\ion{H}{2}}}\xspace}

%\input{aamacros.tex}

\pdfminorversion=4


%%%%%%%%%%%%%%%%%%%%%%%%%%%%%%%%%%%%%%%%%%%%%%%%%%%%%%%%%%%%%%%%
%%%%%%%%%%%  see documentation for information about  %%%%%%%%%%
%%%%%%%%%%%  the options (11pt, defaultstyle, etc.)   %%%%%%%%%%
%%%%%%%  http://www.colorado.edu/its/docs/latex/thesis/  %%%%%%%
%%%%%%%%%%%%%%%%%%%%%%%%%%%%%%%%%%%%%%%%%%%%%%%%%%%%%%%%%%%%%%%%
%		\documentclass[typewriterstyle]{thesis}
% 		\documentclass[modernstyle]{thesis}
% 		\documentclass[modernstyle,11pt]{thesis}
%	 	\documentclass[modernstyle,12pt]{thesis}

%%%%%%%%%%%%%%%%%%%%%%%%%%%%%%%%%%%%%%%%%%%%%%%%%%%%%%%%%%%%%%%%
%%%%%%%%%%%    load any packages which are needed    %%%%%%%%%%%
%%%%%%%%%%%%%%%%%%%%%%%%%%%%%%%%%%%%%%%%%%%%%%%%%%%%%%%%%%%%%%%%
\usepackage{latexsym}		% to get LASY symbols
\usepackage{graphicx}		% to insert PostScript figures
%\usepackage{deluxetable}
\usepackage{rotating}		% for sideways tables/figures
\usepackage{natbib}  % Requires natbib.sty, available from http://ads.harvard.edu/pubs/bibtex/astronat/
\usepackage{savesym}
%\usepackage{pdflscape}
\usepackage{amssymb}
\usepackage{morefloats}
%\savesymbol{singlespace}
\savesymbol{doublespace}
%\usepackage{wrapfig}
%\usepackage{setspace}
\usepackage{xspace}
\usepackage{color}
%\usepackage{multicol}
\usepackage{mdframed}
\usepackage{url}
\usepackage{subfigure}
%\usepackage{emulateapj}
%\usepackage{lscape}
\usepackage{grffile}
\usepackage{import}
\usepackage[utf8]{inputenc}
%\usepackage{longtable}
\usepackage{booktabs}
%\usepackage[yyyymmdd,hhmmss]{datetime}
\usepackage{fancyhdr}
%\usepackage[colorlinks=true,citecolor=blue,linkcolor=cyan]{hyperref}

\usepackage[hang,flushmargin]{footmisc}
\usepackage{ifpdf}


%\usepackage{standalone}
%\standalonetrue






\newcommand{\paa}{Pa\ensuremath{\alpha}}
\newcommand{\brg}{Br\ensuremath{\gamma}}
\newcommand{\msun}{\ensuremath{M_{\odot}}\xspace}			%  Msun
\newcommand{\mdot}{\ensuremath{\dot{M}}\xspace}
\newcommand{\lsun}{\ensuremath{L_{\odot}}\xspace}			%  Lsun
\newcommand{\rsun}{\ensuremath{R_{\odot}}\xspace}			%  Rsun
\newcommand{\lbol}{\ensuremath{L_{\mathrm{bol}}\xspace}}	%  Lbol
\newcommand{\ks}{K\ensuremath{_{\mathrm{s}}}}		%  Ks
\newcommand{\hh}{\ensuremath{\textrm{H}_{2}}\xspace}			%  H2
\newcommand{\dens}{\ensuremath{n(\hh) [\percc]}\xspace}
\newcommand{\formaldehyde}{\ensuremath{\textrm{H}_2\textrm{CO}}\xspace}
\newcommand{\formamide}{\ensuremath{\textrm{NH}_2\textrm{CHO}}\xspace}
\newcommand{\formaldehydeIso}{\ensuremath{\textrm{H}_2~^{13}\textrm{CO}}\xspace}
\newcommand{\methanol}{\ensuremath{\textrm{CH}_3\textrm{OH}}\xspace}
\newcommand{\ortho}{\ensuremath{\textrm{o-H}_2\textrm{CO}}\xspace}
\newcommand{\para}{\ensuremath{\textrm{p-H}_2\textrm{CO}}\xspace}
\newcommand{\oneone}{\ensuremath{1_{1,0}-1_{1,1}}\xspace}
\newcommand{\twotwo}{\ensuremath{2_{1,1}-2_{1,2}}\xspace}
\newcommand{\threethree}{\ensuremath{3_{1,2}-3_{1,3}}\xspace}
\newcommand{\threeohthree}{\ensuremath{3_{0,3}-2_{0,2}}\xspace}
\newcommand{\threetwotwo}{\ensuremath{3_{2,2}-2_{2,1}}\xspace}
\newcommand{\threetwoone}{\ensuremath{3_{2,1}-2_{2,0}}\xspace}
\newcommand{\fourtwotwo}{\ensuremath{4_{2,2}-3_{1,2}}\xspace} % CH3OH 218.4 GHz
\newcommand{\methylcyanide}{\ensuremath{\textrm{CH}_{3}\textrm{CN}}\xspace}
\newcommand{\ketene}{\ensuremath{\textrm{H}_{2}\textrm{CCO}}\xspace}
\newcommand{\ethylcyanide}{\ensuremath{\textrm{CH}_3\textrm{CH}_2\textrm{CN}}\xspace}
\newcommand{\cyanoacetylene}{\ensuremath{\textrm{HC}_{3}\textrm{N}}\xspace}
\newcommand{\methylformate}{\ensuremath{\textrm{CH}_{3}\textrm{OCHO}}\xspace}
\newcommand{\dimethylether}{\ensuremath{\textrm{CH}_{3}\textrm{OCH}_{3}}\xspace}
\newcommand{\gaucheethanol}{\ensuremath{\textrm{g-CH}_3\textrm{CH}_2\textrm{OH}}\xspace}
\newcommand{\acetone}{\ensuremath{\left[\textrm{CH}_{3}\right]_2\textrm{CO}}\xspace}
\newcommand{\methyleneamidogen}{\ensuremath{\textrm{H}_{2}\textrm{CN}}\xspace}
\newcommand{\Rone}{\ensuremath{\para~S_{\nu}(\threetwoone) / S_{\nu}(\threeohthree)}\xspace}
\newcommand{\Rtwo}{\ensuremath{\para~S_{\nu}(\threetwotwo) / S_{\nu}(\threetwoone)}\xspace}
\newcommand{\JKaKc}{\ensuremath{J_{K_a K_c}}}
\newcommand{\water}{H$_{2}$O\xspace}		%  H2O
\newcommand{\feii}{\ion{Fe}{ii}\xspace}		%  FeII

\newcommand{\uchii}{\ion{UCH}{ii}\xspace}
\newcommand{\UCHII}{\ion{UCH}{ii}\xspace}
\newcommand{\hchii}{\ion{HCH}{ii}\xspace}
\newcommand{\HCHII}{\ion{HCH}{ii}\xspace}
\newcommand{\hii}{\ion{H}{ii}\xspace}

\newcommand{\hi}{H~{\sc i}\xspace}
\newcommand{\Hii}{\hii}
\newcommand{\HII}{\hii}
\newcommand{\Xform}{\ensuremath{X_{\formaldehyde}}}
\newcommand{\kms}{\textrm{km~s}\ensuremath{^{-1}}\xspace}	%  km s-1
\newcommand{\nsample}{456\xspace}
\newcommand{\CFR}{5\xspace} % nMPC / 0.25 / 2 (6 for W51 once, 8 for W51 twice) REFEDIT: With f_observed=0.3, becomes 3/2./0.3 = 5
\newcommand{\permyr}{\ensuremath{\mathrm{Myr}^{-1}}\xspace}
\newcommand{\pers}{\ensuremath{\mathrm{s}^{-1}}\xspace}
\newcommand{\perspc}{\ensuremath{\mathrm{pc}^{-2}}\xspace}
\newcommand{\tsuplim}{0.5\xspace} % upper limit on starless timescale
\newcommand{\ncandidates}{18\xspace}
\newcommand{\mindist}{8.7\xspace}
\newcommand{\rcluster}{2.5\xspace}
\newcommand{\ncomplete}{13\xspace}
\newcommand{\middistcut}{13.0\xspace}
\newcommand{\nMPC}{3\xspace} % only count W51 once.  W51, W49, G010
\newcommand{\obsfrac}{30}
\newcommand{\nMPCtot}{10\xspace} % = nmpc / obsfrac
\newcommand{\nMPCtoterr}{6\xspace} % = sqrt(nmpc) / obsfrac
\newcommand{\plaw}{2.1\xspace}
\newcommand{\plawerr}{0.3\xspace}
\newcommand{\mmin}{\ensuremath{10^4~\msun}\xspace}
%\newcommand{\perkmspc}{\textrm{per~km~s}\ensuremath{^{-1}}\textrm{pc}\ensuremath{^{-1}}\xspace}	%  km s-1 pc-1
\newcommand{\kmspc}{\textrm{km~s}\ensuremath{^{-1}}\textrm{pc}\ensuremath{^{-1}}\xspace}	%  km s-1 pc-1
\newcommand{\sqcm}{cm$^{2}$\xspace}		%  cm^2
\newcommand{\percc}{\ensuremath{\textrm{cm}^{-3}}\xspace}
\newcommand{\perpc}{\ensuremath{\textrm{pc}^{-1}}\xspace}
\newcommand{\persc}{\ensuremath{\textrm{cm}^{-2}}\xspace}
\newcommand{\persr}{\ensuremath{\textrm{sr}^{-1}}\xspace}
\newcommand{\peryr}{\ensuremath{\textrm{yr}^{-1}}\xspace}
\newcommand{\perkmspc}{\textrm{km~s}\ensuremath{^{-1}}\textrm{pc}\ensuremath{^{-1}}\xspace}	%  km s-1 pc-1
\newcommand{\perkms}{\textrm{per~km~s}\ensuremath{^{-1}}\xspace}	%  km s-1 
\newcommand{\um}{\ensuremath{\mu \textrm{m}}\xspace}    % micron
\newcommand{\microjy}{\ensuremath{\mu\textrm{Jy}}\xspace}    % micron
\newcommand{\mum}{\um}
\newcommand{\htwo}{\ensuremath{\textrm{H}_2}}
\newcommand{\Htwo}{\ensuremath{\textrm{H}_2}}
\newcommand{\HtwoO}{\ensuremath{\textrm{H}_2\textrm{O}}}
\newcommand{\htwoo}{\ensuremath{\textrm{H}_2\textrm{O}}}
\newcommand{\ha}{\ensuremath{\textrm{H}\alpha}}
\newcommand{\hb}{\ensuremath{\textrm{H}\beta}}
\newcommand{\so}{SO~\ensuremath{5_6-4_5}\xspace}
\newcommand{\SO}{SO~\ensuremath{1_2-1_1}\xspace}
\newcommand{\ammonia}{NH\ensuremath{_3}\xspace}
\newcommand{\twelveco}{\ensuremath{^{12}\textrm{CO}}\xspace}
\newcommand{\thirteenco}{\ensuremath{^{13}\textrm{CO}}\xspace}
\newcommand{\ceighteeno}{\ensuremath{\textrm{C}^{18}\textrm{O}}\xspace}
\def\ee#1{\ensuremath{\times10^{#1}}}
\newcommand{\degrees}{\ensuremath{^{\circ}}}
% can't have \degree because I'm getting a degree...
\newcommand{\lowirac}{800}
\newcommand{\highirac}{8000}
\newcommand{\lowmips}{600}
\newcommand{\highmips}{5000}
\newcommand{\perbeam}{\ensuremath{\textrm{beam}^{-1}}\xspace}
\newcommand{\ds}{\ensuremath{\textrm{d}s}}
\newcommand{\dnu}{\ensuremath{\textrm{d}\nu}}
\newcommand{\dv}{\ensuremath{\textrm{d}v}}
\def\secref#1{Section \ref{#1}}
\def\eqref#1{Equation \ref{#1}}
\def\facility#1{#1}
%\newcommand{\arcmin}{'}

\newcommand{\necluster}{Sh~2-233IR~NE}
\newcommand{\swcluster}{Sh~2-233IR~SW}
\newcommand{\region}{IRAS 05358}

\newcommand{\nwfive}{40}
\newcommand{\nouter}{15}

\newcommand{\vone}{{\rm v}1.0\xspace}
\newcommand{\vtwo}{{\rm v}2.0\xspace}
\newcommand\mjysr{\ensuremath{{\rm MJy~sr}^{-1}}}
\newcommand\jybm{\ensuremath{{\rm Jy~bm}^{-1}}}
\newcommand\nbolocat{8552\xspace}
\newcommand\nbolocatnew{548\xspace}
\newcommand\nbolocatnonew{8004\xspace} % = nbolocat-nbolocatnew
%\renewcommand\arcdeg{\mbox{$^\circ$}\xspace} 
%\renewcommand\arcmin{\mbox{$^\prime$}\xspace} 
%\renewcommand\arcsec{\mbox{$^{\prime\prime}$}\xspace} 

\newcommand{\todo}[1]{\textcolor{red}{#1}}
\newcommand{\okinfinal}[1]{{#1}}
%% only needed if not aastex
%\newcommand{\keywords}[1]{}
%\newcommand{\email}[1]{}
%\newcommand{\affil}[1]{}


%aastex hack
%\newcommand\arcdeg{\mbox{$^\circ$}}%
%\newcommand\arcmin{\mbox{$^\prime$}\xspace}%
%\newcommand\arcsec{\mbox{$^{\prime\prime}$}\xspace}%

%\newcommand\epsscale[1]{\gdef\eps@scaling{#1}}
%
%\newcommand\plotone[1]{%
% \typeout{Plotone included the file #1}
% \centering
% \leavevmode
% \includegraphics[width={\eps@scaling\columnwidth}]{#1}%
%}%
%\newcommand\plottwo[2]{{%
% \typeout{Plottwo included the files #1 #2}
% \centering
% \leavevmode
% \columnwidth=.45\columnwidth
% \includegraphics[width={\eps@scaling\columnwidth}]{#1}%
% \hfil
% \includegraphics[width={\eps@scaling\columnwidth}]{#2}%
%}}%


%\newcommand\farcm{\mbox{$.\mkern-4mu^\prime$}}%
%\let\farcm\farcm
%\newcommand\farcs{\mbox{$.\!\!^{\prime\prime}$}}%
%\let\farcs\farcs
%\newcommand\fp{\mbox{$.\!\!^{\scriptscriptstyle\mathrm p}$}}%
%\newcommand\micron{\mbox{$\mu$m}}%
%\def\farcm{%
% \mbox{.\kern -0.7ex\raisebox{.9ex}{\scriptsize$\prime$}}%
%}%
%\def\farcs{%
% \mbox{%
%  \kern  0.13ex.%
%  \kern -0.95ex\raisebox{.9ex}{\scriptsize$\prime\prime$}%
%  \kern -0.1ex%
% }%
%}%

\def\Figure#1#2#3#4#5{
\begin{figure*}[!htp]
\includegraphics[scale=#4,width=#5]{#1}
\caption{#2}
\label{#3}
\end{figure*}
}

\def\FigureOneCol#1#2#3#4#5{
\begin{figure}[!htp]
\includegraphics[scale=#4,width=#5]{#1}
\caption{#2}
\label{#3}
\end{figure}
}


\def\WrapFigure#1#2#3#4#5#6{
\begin{wrapfigure}{#6}{0.5\textwidth}
\includegraphics[scale=#4,width=#5]{#1}
\caption{#2}
\label{#3}
\end{wrapfigure}
}

% % #1 - filename
% % #2 - caption
% % #3 - label
% % #4 - epsscale
% % #5 - R or L?
% \def\WrapFigure#1#2#3#4#5#6{
% \begin{wrapfigure}[#6]{#5}{0.45\textwidth}
% %  \centercaption
% %  \vspace{-14pt}
%   \epsscale{#4}
%   \includegraphics[scale=#4]{#1}
%   \caption{#2}
%   \label{#3}
% \end{wrapfigure}
% }

\def\RotFigure#1#2#3#4#5{
\begin{sidewaysfigure*}[!htp]
\includegraphics[scale=#4,width=#5]{#1}
\caption{#2}
\label{#3}
\end{sidewaysfigure*}
}

\def\FigureSVG#1#2#3#4{
\begin{figure*}[!htp]
    \def\svgwidth{#4}
    \input{#1}
    \caption{#2}
    \label{#3}
\end{figure*}
}

% originally intended to be included in a two-column paper
% this is in includegraphics: ,width=3in
% but, not for thesis
\def\OneColFigure#1#2#3#4#5{
\begin{figure}[!htpb]
\epsscale{#4}
\includegraphics[scale=#4,angle=#5]{#1}
\caption{#2}
\label{#3}
\end{figure}
}

\def\SubFigure#1#2#3#4#5{
\begin{figure*}[!htp]
\addtocounter{figure}{-1}
\epsscale{#4}
\includegraphics[angle=#5]{#1}
\caption{#2}
\label{#3}
\end{figure*}
}

%\def\FigureTwo#1#2#3#4#5{
%\begin{figure*}[!htp]
%\epsscale{#5}
%\plottwo{#1}{#2}
%\caption{#3}
%\label{#4}
%\end{figure*}
%}

\def\FigureTwo#1#2#3#4#5#6{
\begin{figure*}[!htp]
\subfigure[]{ \includegraphics[scale=#5,width=#6]{#1} }
\subfigure[]{ \includegraphics[scale=#5,width=#6]{#2} }
\caption{#3}
\label{#4}
\end{figure*}
}

\def\FigureTwoAA#1#2#3#4#5#6{
\begin{figure*}[!htp]
\subfigure[]{ \includegraphics[scale=#5,width=#6]{#1} }
\subfigure[]{ \includegraphics[scale=#5,width=#6]{#2} }
\caption{#3}
\label{#4}
\end{figure*}
}

\newenvironment{rotatepage}%
{}{}
   %{\pagebreak[4]\afterpage\global\pdfpageattr\expandafter{\the\pdfpageattr/Rotate 90}}%
   %{\pagebreak[4]\afterpage\global\pdfpageattr\expandafter{\the\pdfpageattr/Rotate 0}}%


\def\RotFigureTwoAA#1#2#3#4#5#6{
\begin{rotatepage}
\begin{sidewaysfigure*}[!htp]
\subfigure[]{ \includegraphics[scale=#5,width=#6]{#1} }
\\
\subfigure[]{ \includegraphics[scale=#5,width=#6]{#2} }
\caption{#3}
\label{#4}
\end{sidewaysfigure*}
\end{rotatepage}
}

\def\RotFigureThreeAA#1#2#3#4#5#6#7{
\begin{rotatepage}
\begin{sidewaysfigure*}[!htp]
\subfigure[]{ \includegraphics[scale=#6,width=#7]{#1} }
\\
\subfigure[]{ \includegraphics[scale=#6,width=#7]{#2} }
\\
\subfigure[]{ \includegraphics[scale=#6,width=#7]{#3} }
\caption{#4}
\label{#5}
\end{sidewaysfigure*}
\end{rotatepage}
\clearpage
}

\def\FigureThreeAA#1#2#3#4#5#6#7{
\begin{figure*}[!htp]
\subfigure[]{ \includegraphics[scale=#6,width=#7]{#1} }
\subfigure[]{ \includegraphics[scale=#6,width=#7]{#2} }
\subfigure[]{ \includegraphics[scale=#6,width=#7]{#3} }
\caption{#4}
\label{#5}
\end{figure*}
}



\def\SubFigureTwo#1#2#3#4#5{
\begin{figure*}[!htp]
\addtocounter{figure}{-1}
\epsscale{#5}
\plottwo{#1}{#2}
\caption{#3}
\label{#4}
\end{figure*}
}

\def\FigureFour#1#2#3#4#5#6{
\begin{figure*}[!htp]
\subfigure[]{ \includegraphics[width=0.5\textwidth]{#1} }
\subfigure[]{ \includegraphics[width=0.5\textwidth]{#2} }
\subfigure[]{ \includegraphics[width=0.5\textwidth]{#3} }
\subfigure[]{ \includegraphics[width=0.5\textwidth]{#4} }
\caption{#5}
\label{#6}
\end{figure*}
}

\def\FigureFourPDF#1#2#3#4#5#6{
\begin{figure*}[!htp]
\subfigure[]{ \includegraphics[width=3in,type=pdf,ext=.pdf,read=.pdf]{#1} }
\subfigure[]{ \includegraphics[width=3in,type=pdf,ext=.pdf,read=.pdf]{#2} }
\subfigure[]{ \includegraphics[width=3in,type=pdf,ext=.pdf,read=.pdf]{#3} }
\subfigure[]{ \includegraphics[width=3in,type=pdf,ext=.pdf,read=.pdf]{#4} }
\caption{#5}
\label{#6}
\end{figure*}
}

\def\FigureThreePDF#1#2#3#4#5{
\begin{figure*}[!htp]
\subfigure[]{ \includegraphics[width=3in,type=pdf,ext=.pdf,read=.pdf]{#1} }
\subfigure[]{ \includegraphics[width=3in,type=pdf,ext=.pdf,read=.pdf]{#2} }
\subfigure[]{ \includegraphics[width=3in,type=pdf,ext=.pdf,read=.pdf]{#3} }
\caption{#4}
\label{#5}
\end{figure*}
}

\def\Table#1#2#3#4#5{
%\renewcommand{\thefootnote}{\alph{footnote}}
\begin{table}
\caption{#2}
\label{#3}
    \begin{tabular}{#1}
        \hline\hline
        #4
        \hline
        #5
        \hline
    \end{tabular}
\end{table}
%\renewcommand{\thefootnote}{\arabic{footnote}}
}


%\def\Table#1#2#3#4#5#6{
%%\renewcommand{\thefootnote}{\alph{footnote}}
%\begin{deluxetable}{#1}
%\tablewidth{0pt}
%\tabletypesize{\footnotesize}
%\tablecaption{#2}
%\tablehead{#3}
%\startdata
%\label{#4}
%#5
%\enddata
%\bigskip
%#6
%\end{deluxetable}
%%\renewcommand{\thefootnote}{\arabic{footnote}}
%}

%\def\tablenotetext#1#2{
%\footnotetext[#1]{#2}
%}

% \def\LongTable#1#2#3#4#5#6#7#8{
% % required to get tablenotemark to work: http://www2.astro.psu.edu/users/stark/research/psuthesis/longtable.html
% \renewcommand{\thefootnote}{\alph{footnote}}
% \begin{longtable}{#1}
% \caption[#2]{#2}
% \label{#4} \\
% 
%  \\
% \hline 
% #3 \\
% \hline
% \endfirsthead
% 
% \hline
% #3 \\
% \hline
% \endhead
% 
% \hline
% \multicolumn{#8}{r}{{Continued on next page}} \\
% \hline
% \endfoot
% 
% \hline 
% \endlastfoot
% #7 \\
% 
% #5
% \hline
% #6 \\
% 
% \end{longtable}
% \renewcommand{\thefootnote}{\arabic{footnote}}
% }

\def\TallFigureTwo#1#2#3#4#5#6{
\begin{figure*}[htp]
\epsscale{#5}
\subfigure[]{ \includegraphics[width=#6]{#1} }
\subfigure[]{ \includegraphics[width=#6]{#2} }
\caption{#3}
\label{#4}
\end{figure*}
}

		% file containing author's macro definitions
%%% This file is generated by the Makefile.
\newcommand{\githash}{3bc668e}\newcommand{\gitdate}{2017-08-04\xspace}\newcommand{\gitauthor}{Adam Ginsburg (keflavich)\xspace}

\newcommand{\percent}{\%\xspace}

\begin{document}
\title{FeO}
\titlerunning{FeO}
\authorrunning{Ginsburg et al}
\begin{abstract}
We report the nondetection of FeO in the Orion `Bullets'.
\end{abstract}

\section{Introduction}

Iron is among the most abundant atoms in the universe, with a greater abundance
than any of the other heavy elements (the \emph{real} metals).  However,
despite a long search history \citep{Merer1982a}, it has only once been
detected in molecular form in space, toward the Sgr B2 Molecular Heimat
\citep{Walmsley2002a,Furuya2003a}.  Most likely, this is because iron in all of
its forms is readily incorporated into dust grains, from which it is very
difficult to remove \citep{Silvia2012a,Nozawa2003a}.  One plausible means
of destroying dust grains, thereby releasing iron into the gas phase, is
via strong shocks in protostellar outflows.  If produced this way, gas-phase
iron could prove a powerful tracer of protostellar outflows.

Other molecules are used as such tracers at present, such as SiO.  Silicon is
also a reasonably abundant metal (and here we abuse the term as astronomers),
but it is frequently detected in interstellar gas in its oxide, SiO.  In most
quiescent clouds, it is not detected, and is presumably locked into grains
\citep{Ziurys1989a}.  However, it is frequently seen in outflows and PDRs, as
it is liberated from the grains in moderate- to high-velocity shocks
\citep[$v_s>10$
\kms;][]{Schilke1997a,Schilke2001a,Jimenez-Serra2008a,Anderl2013a,Gusdorf2008a,Gusdorf2008b}.
While this explanation has led to the acceptance of SiO as a shock tracer,
recent observations of diffuse SiO in infrared-dark clouds
\citep{Jimenez-Serra2010a} and throughout the entire central molecular zone of
our Galaxy \citep{Jones2012a} imply that SiO is not uniquely a tracer of the
high-velocity shocks in which it is most readily seen.

Iron oxide (FeO) may provide a better tracer of strong shocks.  Evidently, iron
rarely exists in the gas phase.  Most likely, this means that iron is trapped
within dust grains.  When iron is released from dust, it will form FeO under
some conditions (though it may also be directly released as FeO).  In order for
FeO to be produced in the gas phase from atomic iron, Fe must undergo
endothermic reactions with O$_2$ ($T\sim10^4$ K) or OH ($T\sim1550$ K)
\citep{Walmsley2002a}. Such high temperatures are present only in high-velocity
shocks, which in principle makes FeO more uniquely a tracer of such shocks
than SiO.

%Given iron's low abundance outside of shocks, even in HII regions, it must
%require strong sputtering or shattering to be released into the gas phase.

The physics of sputtering in shocks is incompletely understood.  The
structure of the grains may play an important role in which species can be
expelled from them.  The standard assumption is to divide grains into
``cores'', which are the most likely to contain iron-bearing molecules, and
``mantles'' that include ices that will evaporate at high temperatures.  The
cores are much more difficult to disrupt, requiring higher shock velocities.
Iron's low gas-phase abundance hints that it may be preferentially located in grain
cores.  Examination of where grain cores are disrupted can provide useful
constraints on the efficiency of sputtering, which presently is subject to a
great deal of guesswork \citep{Schilke1997a}.  An understanding of grain
destruction physics in a pristine environment - one unaffected by simultaneous
grain formation - is essential to constrain models of dust formation and
destruction in supernovae.

FeO may trace dust within jets launched from low-mass accreting stars.
Near-infrared observations have detected ionized [Fe II] along Herbig-Haro
flows \citep[e.g. HH 111;][]{Nisini2002a}, but the iron mass budget in these
flows is unconstrained.  If jets are launched close to the star (inside the
dust destruction radius) we might expect Fe to be in the gas phase immediately
after launch, otherwise the iron must be entrained along the flow. Following
how FeO changes with distance from the star should provide clues about where
the jet launching region is.   

We were therefore motivated to search for FeO in a region with high-velocity
shocks.  We chose part of the Orion outflow as our target.


The Orion KL nebula contains one of the most spectacular explosive outflows on
the sky.  As shown in Figure \ref{fig:targetbullets}
\citep{Bally2015a,Bally2017a}, the nebula is full of extremely bright [Fe II]
emission knots, indicating unambiguously that iron is present in the gas phase.  
%In order to get to that phase, it most likely was
%sputtered from dust in the same high-velocity ($v_s\gtrsim100$ \kms) shocks
%that ionize and excite most of the [Fe II] emission.
The Orion bullets provide a more favorable region to find uniquely
shock-tracing species than either the inner Orion KL nebula or the Sgr B2 hot cores,
both of which suffer from extremely high line densities and both spatial and
spectral confusion \citep{Niederhofer2012a,Belloche2013a}.

The expanding thin shells of excited \hh around the bullets indicate that it is
the surrounding medium, not the bullet material, that is being excited along
the bullet wakes: any molecular material from the bullets would already have
been dissociated in the bow shock or in the postshock hot plasma.  The [Fe II]
emission is therefore also most likely coming from the ISM gas being plowed
into by the bullets.  

Previous searches in Orion have been done with single dish telescopes with
large beams \citep[40\arcsec;][]{Merer1982a}, but the bullets have angular
scales $\sim1-3$ arcseconds, implying a maximum area filling factor
$\lesssim0.06\%$.  In this work, we present a search that is about 1000 times more
sensitive than previous observations (assuming the above filling factor),
but we did not detect FeO.

In Section \ref{sec:observations}, we describe the observational setup.


\section{Observations}

We observed the northern Orion bullets with ALMA in bands 3 and 4.
The observing setup is described in Table \ref{tab:observations}.
Our spectral setup included 8 windows in Band 3 and 10 in Band 4;
the spectral setup is described in Table \ref{tab:lines}.


\begin{table*}[htp]
\centering
\caption{Observation Summary}
\begin{tabular}{lllll}
\label{tab:observations}
Date & Array & Observation Duration &  Baseline Length Range  & \# of antennae\\
     &       & seconds              & meters                    & \\
\hline

02-Aug-2016 & 12m & 5029 & 15-1396 & 39\\\\
22-Aug-2016 & 12m & 3348 & 15-1462 & 40\\\\
09-May-2017 & 12m & 2424 & 17-1124 & 43\\

\hline
\end{tabular}
\end{table*}


\begin{table*}[htp]
\begin{tabular}{ccccc}
\hline \hline
Species & Quantum Numbers & Frequency & band & SPW ID \\
 &  & $\mathrm{GHz}$ &  &  \\
\hline
CH$_3$CCHv=0 & \ensuremath{9_{4}-8_{4}} & 153.77022 & 4.0 & 43.0 \\
CH$_3$CCHv=0 & \ensuremath{9_{3}-8_{3}} & 153.79077 & 4.0 & 43.0 \\
CH$_3$CCHv=0 & \ensuremath{9_{2}-8_{2}} & 153.80546 & 4.0 & 43.0 \\
CH$_3$CCHv=0 & \ensuremath{9_{1}-8_{1}} & 153.81427 & 4.0 & 43.0 \\
CH$_3$CCHv=0 & \ensuremath{9_{0}-8_{0}} & 153.81721 & 4.0 & 43.0 \\
CCS & \ensuremath{8,7-7,6} & 93.8701 & 3.0 & 25.0 \\
CH$_3$OHvt=0 & \ensuremath{0_{0,0}-1(-1,1)} & 108.89393 & 3.0 & 31.0 \\
HC$_3$Nv=0 & \ensuremath{J=12-11} & 109.17364 & 3.0 & 35.0 \\
SO~$^3\Sigma$v=0 & \ensuremath{2_{3}-1_{2}} & 109.25221 & 3.0 & 35.0 \\
CCS & \ensuremath{N=12-11,J=11-10} & 153.44977 & 4.0 & 43.0 \\
HNCOv=0 & \ensuremath{7_{0,7}-6_{0,6},F=6-7} & 153.86499 & 4.0 & 45.0 \\
FeO & \ensuremath{3-2_1-+} & 93.488409 & 3.0 & 27.0 \\
FeO & \ensuremath{3-2_1+-} & 93.49267 & 3.0 & 27.0 \\
FeO & \ensuremath{3-2_0+-} & 93.999341 & 3.0 & 25.0 \\
FeO & \ensuremath{3-2_0-+} & 94.002772 & 3.0 & 25.0 \\
FeO & \ensuremath{5-4_0+-} & 156.657992 & 4.0 & 39.0 \\
FeO & \ensuremath{5-4_0-+} & 156.663754 & 4.0 & 39.0 \\
FeO & \ensuremath{5-4_1-+} & 155.806807 & 4.0 & 41.0 \\
FeO & \ensuremath{5-4_1+-} & 155.813777 & 4.0 & 41.0 \\
\hline
\end{tabular}
\end{table*}


We achieved a continuum sensitivity varying from 0.04-0.2 mJy depending on the
weighting scheme used to produce the continuum images (see Table
\ref{tab:contsensitivity}).

\begin{table*}[htp]
\begin{tabular}{ccccccc}
\label{tab:contsensitivity}
Band & Robust & RMS & BMAJ & BMIN & BPA & K/Jy \\
 &  & $\mathrm{mJy}$ &  &  &  &  \\
\hline
3 & -2 & 0.17 & 0.58 & 0.43 & -80.2 & 480 \\
3 & 0 & 0.035 & 0.61 & 0.46 & -77.6 & 430 \\
3 & 2 & 0.091 & 0.79 & 0.64 & -73.4 & 240 \\
4 & -2 & 0.37 & 0.34 & 0.28 & 71.0 & 550 \\
4 & 0 & 0.066 & 0.36 & 0.3 & 68.2 & 490 \\
4 & 2 & 0.14 & 0.44 & 0.38 & 40.0 & 310 \\
\hline
\end{tabular}
\end{table*}


\section{Results}
\subsection{Lines}
We report only nondetections of FeO.  We searched the cubes channel-by-channel
and performed spatial averages over apertures approximately matching the
near-infrared \hh and [Fe II] images.  The spatial averages were performed
on each of the robust-weighted -2, 0, and 2 images.

We detected emission in the \cyanoacetylene J=12-11 and CH$_3$CCH 9-8 lines,
but this emission was highly extended and not associated with outflow features.



\end{document}
